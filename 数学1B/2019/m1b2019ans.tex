\documentclass[dvipdfmx]{jsarticle}
\usepackage{amsmath, amssymb, type1cm, amsfonts, latexsym, mathtools, bm, amsthm, url, pict2e, otf}
\usepackage{tikz}
\usepackage{newpxtext, newpxmath}
\usepackage{graphicx}
\usetikzlibrary{intersections,calc,arrows.meta}
\begin{document}
  \section{}
    \subsection{}
      \begin{align*}
        \lim_{x \to \infty} f(x) = \lim_{x \to \infty} \mathrm{Tan}^{-1} x + \lim_{x \to \infty} \mathrm{Tan}^{-1} \frac{1}{x}\\
        = \frac{\pi}{2} + 0 = \underline{\frac{\pi}{2}}
      \end{align*}
    \subsection{}
      $x = \tan y$の両辺を$x$で微分すると
      \begin{align*}
        1 = \frac{1}{\frac{1}{1 + \tan^2 y}} \cdot \frac{\mathrm{d}y}{\mathrm{d}x}\\
        \Leftrightarrow \frac{\mathrm{d}y}{\mathrm{d}x} = \frac{1}{1 +\tan^2 y} = \frac{1}{1 + x^2}
      \end{align*}
      これより,
      \begin{align*}
        f'(x) = \frac{1}{1 + x^2} + \frac{1}{1+\left(\frac{1}{x}\right)^2}\cdot -\left(\frac{1}{x}\right)^2\\
        = \underline{0}
      \end{align*}
    \subsection{}
      $b/a > 0$の時,
      \begin{align*}
        \mathrm{Tan}^{-1} \left(\frac{b}{a}\right) + \mathrm{Tan}^{-1} \left(\frac{a}{b}\right) = \frac{\pi}{2}
      \end{align*}
      である.これは正接の定義による.
      このことを用いると
      $f(1) = f(2) = \underline{\pi / 2}$
  \section{}
      $x^3 + 3x^2 + 4x + 2 = (x + 2)(x^2 + 2x + 2)$より
      \begin{align*}
        \int_{1}^{2} \frac{1}{x^3 + 3x^2 + 4x + 2} \mathrm{d}x 
        = \int_{1}^{2} \left(\frac{1}{x + 1} - \frac{x + 1}{x^2 + 2x + 2}\right) \mathrm{d}x\\
        = \left[\log |x + 1|\right]_{1}^{2} -\frac{1}{2}\int_{1}^{2} \frac{\left(\frac{(x+1)^2 + 1}{2}\right)'}{(x+1)^2 + 1}\mathrm{d}x
        = \log \frac{3}{2} - \frac{1}{2} \left[\log \frac{(x+1)^2 + 1}{2}\right]_{1}^{2}\\
        = \underline{\log \frac{3}{2} - \frac{1}{2}\log 2}
      \end{align*}
  \section{}
    \subsection{}
      積分区間Aは直線と放物線で囲まれる領域であり,右上図の通りである.
      \begin{tikzpicture}[scale = 3]
          \draw[->,>=stealth,semithick](-0.5,0)--(1.5,0)node[above]{$x$};%x軸
          \draw[->,>=stealth,semithick](0,-0.5)--(0,1.5)node[right]{$y$};%y軸
          \draw(0,0)node[below left]{O};%原点
          \draw[domain=0:1] plot(\x,\x) node[above left]{$y=x$};
          \draw[domain=0:1] plot(\x,\x^2)node[below right]{$y=x^2$};
          \draw[dashed](0, 1)node[left]{1}--(1, 1)--(1, 0)node[below]{1};
      \end{tikzpicture}
      これより,
      \begin{align*}
        \int_{0}^{1} \left(\int_{y}^{\sqrt{y}} f(x, y) \mathrm{d}x\right) \mathrm{d}y = \iint_{A} f(x, y) \mathrm{d}x \mathrm{d}y\\
        = \underline{\int_{0}^{1} \left(\int_{x^2}^{x} f(x, y) \mathrm{d}y\right)} \mathrm{d}x
      \end{align*}
    \subsection{}
      前問と同様に考えれば,
      \begin{align*}
        \int_{0}^{1}\int_{\sqrt{1-y^2}}^{2} \frac{y}{-y^4 + 2y^2+15} \cdot x^3 \mathrm{d}x \mathrm{d}y
        = \frac{1}{4}\int_{0}^{1} \frac{y}{-y^4 + 2y^2+15} \cdot (-y^4 + 2y^2+15) \mathrm{d}y
        = \underline{\frac{1}{8}}
      \end{align*}
  \section{}
    \subsection{}
      $x = r \cos \theta, y = r \sin \theta$ と準備すると,
      \begin{align*}
        J = 
        \begin{vmatrix}
          \cos \theta & -r\sin \theta \\
          \sin \theta & r \sin \theta
        \end{vmatrix}
        = r\\
        f(x, y) = r ^{-3} \sin \left(\frac{1}{r}\right)\\
        \overline{B}(R) - B(3/\pi) = \left\{(r, \theta) \middle| 3/ \pi \leq r \leq R, 0 \leq \theta < 2\pi\right\}
      \end{align*}  
      ここで
      \begin{align*}
        g(R) = \iint_{\overline{B}(R) - B(3/\pi)} r ^{-3} \sin \left(\frac{1}{r}\right) \cdot r \mathrm{d}r \mathrm{d}\theta
        = -\int_{0}^{2\pi} \int_{3/\pi}^{R} \left(\frac{1}{r}\right)' \sin \left(\frac{1}{r}\right) \mathrm{d}r \mathrm{d}\theta\\
        = \int_{0}^{2\pi} \left[\cos \left(\frac{1}{r}\right)\right]_{3/\pi}^{R}\mathrm{d}\theta\\
        = 2\pi \left(\cos \left(\frac{1}{R}\right) - \frac{1}{2}\right)
      \end{align*}
      より問題文の条件の時,$g(R)$は$R$に関して単調増加である.これは,外側の円周の半径が大きくなることに対応しているため,
      \begin{align*}
        \overline{B}(R) - B(3/\pi) \subset D(R) - B(3/\pi) \subset \overline{B}(\sqrt{2}R) - B(3/\pi)
      \end{align*}
      から,題意が成立する.\qed
      %どう書けばいいの?
    \subsection{}
      前問で
      \begin{align*}
        \lim_{R \to \infty} g(R) = \pi, \lim_{R \to \infty} g(\sqrt{2}R) = \pi
      \end{align*}
      より,挟み撃ちの定理から
      \begin{align*}
        \lim_{R \to \infty} \iint_{D(R) - B(3/\pi)} f(x, y) \mathrm{d}x\mathrm{d}y = \underline{\pi}
      \end{align*}
  \section{}
    \subsection{}
      \begin{align*}
        \nabla \bm{r}(t) =
        \begin{pmatrix}
          e^t(\cos t - \sin t)\\
          e^t(\sin t - \cos t)\\
          e^t
        \end{pmatrix}
        , \bm{f}(x, y, z) = \frac{1}{2}
        \begin{pmatrix}
          e^{-t} \cos t\\
          e^{-t} \sin t\\
          e^{-t}
        \end{pmatrix}
      \end{align*}
      を用いれば
      \begin{align*}
        \int_{\Gamma} \bm{f} \cdot \mathrm{d} \bm{r} =\frac{1}{2} \int_{0}^{10} (\cos t (\cos t - \sin t) + \sin t (\sin t - \cos t) + 1) \mathrm{d}t\\
        = \int_{0}^{10} 1 - \sin t (\sin t )' \mathrm{d}t
        = [t - \frac{\sin^2 t}{2}]_{0}^{10}\\
        = \underline{10 - \frac{\sin^2 10}{2}}
      \end{align*}
    \subsection{}
      グラフ表示されることから,$\bm{r}(x, y) = (x, y, \phi(x, y))$とすると,
      \begin{align*}
        \phi_x = \frac{x}{\sqrt{1 - x^2 - y^2}},
        \phi_y = \frac{y}{\sqrt{1 - x^2 - y^2}}\\
        \partial_x \bm{r} = (1, 0, \phi_x),
        \partial_y \bm{r} = (0, 1, \phi_y)\\
        \bm{n} = \pm \frac{\partial_x \bm{r} \times \partial_y \bm{r}}{\|\partial_x \bm{r} \times \partial_y \bm{r}\|}\\
        = \pm \frac{1}{\sqrt{\phi^2_x + \phi^2_y + 1}} \cdot 
        \begin{pmatrix}
          -\phi_x\\
          -\phi_y\\
          1
        \end{pmatrix}
      \end{align*}
      ここで,$\bm{n} \cdot \bm{e}_z \geq 0$より複号は正で,$\phi_x, \phi_y$にそれぞれ代入すれば,
      \begin{align*}
        \bm{n} =\underline{\begin{pmatrix} -x\\ -y\\ \sqrt{1-x^2-y^2} \end{pmatrix}}
      \end{align*}
      さらに,
      \begin{align*}
        \int_{A} \bm{f} \cdot \bm{n} \mathrm{d}S
        = \int_{0}^{2\pi}\int_{0}^{\sqrt{3}/2} (-r^2 \cos^2 \theta -r^2 \sin^2 \theta + 1 - r^2) \mathrm{d}r \mathrm{d}\theta\\
        = \int_{0}^{2\pi}\int_{0}^{\sqrt{3}/2} (1 - 2r^2) \mathrm{d}r \mathrm{d}\theta
        = \int_{0}^{2\pi} \left[r- \frac{2}{3} r^3\right]_{0}^{\sqrt{3}/2} \mathrm{d}\theta\\
        = \underline{\frac{\sqrt{3}}{2}\pi}
      \end{align*}
\end{document}
